\section{SED Analysis: an overview}
\label{sec:overview}

Fitting spectral energy distributions allows astronomers to estimate fundamental properties like stellar mass, star formation rates, dust content and overall environments of astronomical objects (e.g. \citet{1998AJ....115.1329S}, \citet{2001ApJS..137..139S}, \citet{2007ApJS..169..328R}, and many others). With deeper wide-field surveys and increasing datasets over the years, astronomers have been able to utilize multi-wavelength SEDs more frequently for their research. As such, many robust SED analysis codes have been created to help astronomers model, fit, and derive physical quantities from SEDs \citep{2011Ap&SS.331....1W,2013ARA&A..51..393C}. These widely-used codes implement a diverse set of methods, for instance: inversion (PAHFIT \citep{http://adsabs.harvard.edu/abs/2007ApJ...656..770S}, STARLIGHT \cite{http://adsabs.harvard.edu/abs/2004MNRAS.355..273C}), principal component analysis \citep{2009MNRAS.394.1496B}, and Bayesian inference (GalMC \citep{2011ApJ...737...47A}; VOSA \citep{2008A&A...492..277B}; BPZ \citep{http://adsabs.harvard.edu/abs/2000ApJ...536..571B}); also common are home-grown fitting routines that either tweak existing code or are developed from the ground-up (). 

Most distributed fitting packages are tailored for specific data sets or spectral ranges (e.g. PAHFIT, STARLIGHT), providing robust fitting methods and results. They require the data to be in a specific format with specific units in order for the tool to work properly. When fitting a broadband SED that spans over decades in the spectrum, the astronomer will gather data from different public archives and team members to add to their own dataset. More often than not, the datasets are presented in different file formats and units. The user must provide their own methods to extract the necessary data from each file, convert the units, and output a file in the single format supported by the tool. 

%The astronomer may also want to inspect the SED, for example plotting it against different units, normalizing some of points or spectral segments, shifting the SED to another redshift, and performing other simple visualization tasks.

While SED analysis tools have input formats that may be different from each other, they nevertheless require effectively the same information to run. It thus becomes a matter of providing an interoperable framework that makes building, viewing, and analyzing SEDs a straightforward process.

% However, most SED analysis tools require similar data and pre-processing steps, like gathering and converting the data to compliant formats for the tools. Thus it becomes only a matter of providing an interoperable framework that makes building, viewing, and analyzing SEDs a straightforward process.

Following VO efforts to seamlessly combine data services and applications, Iris offers a standard means of building large broadband SEDs from different sources in various data formats, while providing robust fitting methods and interactive visualization capabilities. Users can input SED data from a local file or a URL, with high leniency on the data format. Users may also beam data from other VO-enabled applications or data archive services through SAMP (Simple Application Messaging Protocol; \citet{2011arXiv1110.0528T}). If the data format follows the IVOA Spectrum Data Model v1.03 \citep{2012arXiv1204.3055M}, or in other words a VO-compliant VOTable or FITS file, the data is read-in without any input by the user; otherwise, the user supplies the units and  mapping to the spectral-flux coordinates in the file. How this method works is described in Section \ref{sec:components}.

%While each code is tailored for specific datasets or object types, most SED analysis tools share similar necessities and require the same pre-processing steps. Thus it is worth developing an interoperable SED tool that Following VO efforts to seamlessly combine data services and applications, Iris offers a standard means of building large broadband SEDs from different sources in various data formats, while providing robust fitting methods and interactive visualization capabilities.

%[FIX ME: make me a happy, positive statement] While these exacting steps are worthwhile for the quality fitting results, gathering, pre-processing, and converting data to plug it in to a fitting engine should be standard/interpoerable and straightfoward. 

%This inconvenience -- building SEDs from multiple sources -- drove a part of Iris' SED analysis design. Following VO efforts to seamlessly combine data services and applications, Iris offers a standard means of building large broadband SEDs from different sources in various data formats, while providing robust fitting methods and interactive visualization capabilities. Users can input SED data from a local file or a URL, with high leniency on the data format. Users may also beam data from other VO-enabled applications or data archive services through SAMP (Simple Application Messaging Protocol; \citet{2011arXiv1110.0528T}). If the data format follows the IVOA Spectrum Data Model v1.03 \citep{2012arXiv1204.3055M}, or in other words a VO-compliant VOTable or FITS file, the data is read-in without any input by the user; otherwise, the user supplies the units and  mapping to the spectral-flux coordinates in the file. How this method works is described in Section \ref{sec:components}.

Iris is meant to be an open box SED analysis tool. If Iris is missing a certain functionality, the user may develop a plugin for that functionality and add it to Iris. Thus, a user can employ Iris' SED building capabilites while running custom analysis, which in theory could be one of the fitting packages discussed above. We explore this possibility in detail in Section \ref{sec:plugins}

%Users can input SED data from a local file, a URL, another VO-enabled application, or directly from VO-enabled data archive services, with high leniency on the data format.

%\begin{comment}
%Following the VO effort for seamless interoperability between data services and applications,  

%The Virtual Observatory is an effort to standardize data formats and services so that users can seamlessly exchange data back and forth between archives and applications.  
%
%Iris offers a standard means of building large broadband SEDs from different sources in various data formats, while providing robust fitting methods and interactive visualization capabilities.
%\end{comment}