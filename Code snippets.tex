\section{Code Snippets}
\subsection{IrisComponent interface}
\label{lst:component}
\begin{verbatim}
/**
 * Copyright (C) 2012 Smithsonian Astrophysical Observatory
 *
 * Licensed under the Apache License, Version 2.0 (the "License");
 * you may not use this file except in compliance with the License.
 * You may obtain a copy of the License at
 *
 *         http://www.apache.org/licenses/LICENSE-2.0
 *
 * Unless required by applicable law or agreed to in writing, software
 * distributed under the License is distributed on an "AS IS" BASIS,
 * WITHOUT WARRANTIES OR CONDITIONS OF ANY KIND, either express or implied.
 * See the License for the specific language governing permissions and
 * limitations under the License.
 */

/*
 * To change this template, choose Tools | Templates
 * and open the template in the editor.
 */

package cfa.vo.iris;

import java.util.List;
import org.astrogrid.samp.client.MessageHandler;

/**
 *
 * Interface implemented by Iris components. By implementing this interface the components
 * allow the framework to retrieve the information needed to run and initialize them.
 *
 * @author olaurino
 */
public interface IrisComponent {
    /**
     * This method is invoked to initialize the component. If the component has to launch windows, frames or
     * background services, this is the right method to do so. Otherwise the component will be called only if a callback
     * is invoked.
     *
     * @param app A reference to the running application
     * @param workspace A reference to the application's workspace
     */
    void init(IrisApplication app, IWorkspace workspace);
    /**
     * Return the name of this component. This name might be listed in a widget along with the other registered components.
     * @return The component's name as a String.
     */
    String getName();
    /**
     * Get e description for this component. The description might be listed in a widget along with the other
     * registered components.
     *
     * @return The component's description as a String.
     */
    String getDescription();
    /**
     * Get a command line interface object for this component.
     * @return A CLI object
     */
    ICommandLineInterface getCli();
    /**
     * Initialize the Command Line Application interface
     * @param app Reference to the enclosing application
     */
    void initCli(IrisApplication app);
    /**
     * The component can contribute menu items and desktop buttons to the enclosing GUI applications
     * by providing a list of MenuItems.
     *
     * @return A list of the menu items this component will contribute to the application.
     */
    List<IMenuItem> getMenus();
    /**
     * The component can register any number of SAMP message listeners by providing a list of them.
     *
     * @return A list of the SAMP message listeners that have to be registered to the SAMP hub.
     */
    List<MessageHandler> getSampHandlers();

    /**
     * Callback invoked when the component is shutdown
     */
    void shutdown();
}
\end{verbatim}

\subsection{Command Line Interface} \label{lst:cli}
\begin{verbatim}
/**
 * Copyright (C) 2012 Smithsonian Astrophysical Observatory
 *
 * Licensed under the Apache License, Version 2.0 (the "License");
 * you may not use this file except in compliance with the License.
 * You may obtain a copy of the License at
 *
 *         http://www.apache.org/licenses/LICENSE-2.0
 *
 * Unless required by applicable law or agreed to in writing, software
 * distributed under the License is distributed on an "AS IS" BASIS,
 * WITHOUT WARRANTIES OR CONDITIONS OF ANY KIND, either express or implied.
 * See the License for the specific language governing permissions and
 * limitations under the License.
 */

/*
 * To change this template, choose Tools | Templates
 * and open the template in the editor.
 */

package cfa.vo.iris;

/**
 * A simple interface for providing CLI access in an extensible, pluggable way
 * @author olaurino
 */
public interface ICommandLineInterface {
    /**
     * The name that has to be associated with the implementing component.
     * When the calling application parses the command line, it will interpret the
     * first argument as the component to which the command has to be relayed, using this string
     * as a key.
     *
     * @return The compact name that identifies this CLI
     */
    String getName();
    /**
     * Callback that gets called when a command line is parsed and associated to the implementing component.
     *
     * @param args The command line arguments.
     */
    void call(String[] args);
}
\end{verbatim}

\subsection{User Model Example} \label{lst:user_model_example}
\begin{verbatim}
import numpy as np

def modified_blackbody(p, x):
    """
    Modified blackbody of the form:
    A * B_lambda(T) * (c / (lambda / lambda_0))**beta

    Parameters
    ----------
    p : [lambda_0, A, temp, beta]

        p[0] 'lambda_0' : refernce wavelength
        p[1] 'A' : amplitude of model at reference wavelength
        p[2] 'temp' : temperature of blackbody
        p[3] 'beta' : dust emissivity index

    x : array
        spectral values, in Angstroms
    """
    
    # blackbody function, in terms of wavelength (in Angstroms)
    c1 = 1.438786e8  # in AA K
    efactor = c1 / p[2]
    numerator = p[1] * np.power(p[0], 5.0) * (np.exp(efactor / p[0]) - 1.0)
    denominator = np.power(x, 5.0) * (np.exp(efactor / x) - 1.0)
    B_lambda = numerator / denominator

    # speed of light in Angstroms/second
    c = 2.998e18

    powerlaw = (c / (x/p[0]))**p[3]

    return B_lambda * powerlaw
\end{verbatim}

\subsection{Template Configuration File} \label{lst:templateconfig}
\begin{verbatim}
# INDEX REFER MODELFLAG FILENAME
0.0     5000  1   /home/user/iris-2.0.1-unix-x86_64/examples/sed_temp_index-0.00.dat
-0.10   5000  1   /home/user/iris-2.0.1-unix-x86_64/examples/sed_temp_index-0.10.dat
-0.25   5000  1   /home/user/iris-2.0.1-unix-x86_64/examples/sed_temp_index-0.25.dat
-0.35   5000  1   /home/user/iris-2.0.1-unix-x86_64/examples/sed_temp_index-0.35.dat
-0.50   5000  1   /home/user/iris-2.0.1-unix-x86_64/examples/sed_temp_index-0.50.dat
\end{verbatim}